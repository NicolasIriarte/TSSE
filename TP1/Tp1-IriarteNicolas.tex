\documentclass[
  11pt, % The default document font size, options: 10pt, 11pt, 12pt
  codirector, % Uncomment to add a codirector to the title page
]{charter}
\usepackage{enumitem}
\usepackage{pdflscape}
\usepackage{tikz}
%\usetikzlibrary{shapes,arrows}
%\usepackage{tikz}
\usetikzlibrary{positioning, arrows.meta, backgrounds, fit}
\usepackage{fontawesome5}
\usepackage{tikz,tkz-tab}
\usepackage{booktabs} % Para tablas más elegantes
\usetikzlibrary{positioning, arrows.meta, backgrounds, fit}

\usetikzlibrary{matrix,arrows, positioning,shadows,shadings,backgrounds, calc, shapes, tikzmark}

\usepackage{fmtcount}
\usepackage{xurl}



% Completar los siguintes Campos
\materia{Testing de Software en Sistemas Embebidos}
\bimestre{cuarto bimestre}
\docentes{Alejandro Permingeat; Esteban	Volentini; Mariano Finochietto y Rafael Oliva}
\titulo{Testing de emulador de microprocesaro Leon3 para desarrollo de software satelital y simuladores}
\posgrado{Carrera de Especialización en Sistemas Embebidos}
\autor{Ing. Iriarte Fernandez, Nicolás Ezequiel
  (NicolasIriarte95@gmail.com)}
\director{Esp. Lic. Horro, Nicolás Eduardo}
\pertenenciaDirector{INVAP.\@S.E.}
\codirector{}
\pertenenciaCoDirector{}
\fechaINICIO{10 de Marzo de 2024}

\begin{document}

\maketitle
\tableofcontents

\newpage

\section*{Registros de cambios}
\label{sec:registro}


\begin{table}[ht]
	\label{tab:registro}
	\centering
	\begin{tabularx}{\linewidth}{@{}|c|X|c|@{}}
		\hline
		\rowcolor[HTML]{C0C0C0}
		Revisión & \multicolumn{1}{c|}{\cellcolor[HTML]{C0C0C0}Detalles de los cambios realizados} & Fecha      \\ \hline
		0      & Creación del documento.                                 &\fechaInicioName \\ \hline

    		1      & Se aplican cambios sugeridos por Salamandri Santiago. & 20 de noviembre de 2023 \\ \hline


    \hline

	\end{tabularx}
	\label{sec:cierre}
\end{table}

%% \pagebreak


\section*{Documentos anexos}
\label{sec:documentos_anexos}

%%%%%%%%%%%%%%%%%%%%%%%%%%%%%%%%%%%%%%%%%%%%%%%%%%
\begin{table}[h!]
	\centering
	\begin{tabular}{ | m{1.5cm} | m{3cm} | m{10.5cm} | }
		\hline
		\rowcolor{gray!50} % Coloring the first row
		\textbf{Ref.} & \textbf{Nombre} & \textbf{Descripción} \\ \hline
    AD.01 & NEMU-SRD-1 & Especificación de requerimientos de software. \\ \hline
		AD.02 & NEMU-UCD-0 & Defenición de casos de uso y arquitectura de software.
    \\ \hline
	\end{tabular}
  \caption{Documentos anexos.}
  \label{tab:referencias}

\end{table}
%%%%%%%%%%%%%%%%%%%%%%%%%%%%%%%%%%%%%%%%%%%%%%%%%%

\pagebreak


\section{Introducción}
\label{sec:org60390fa}

En el presente documento se detallarán todos los aspectos relacionados con la especificación del Master Test Plan referentes al desarrollo del software del ``Emulador de microprocesador Leon3 para desarrollo de software satelital y simuladores'' referenciado en la entrada \textbf{AD.01} en la Tabla~\ref{tab:referencias}.


\subsection{Contenidos}
\label{sec:org434c3ef}

Los contenidos del presente Master Test Plan son:

\begin{itemize}
\item Asignaciones
\item Bases del test
\item Estrategia general del test
\item Estrategia por nivel de prueba
\end{itemize}

\section{Asignaciones}
\label{sec:orgc1c4017}

\subsection{Responsable}
\label{sec:orgdaca22c}

El responsable de la elaboración de este documento es Nicolás Iriarte, ingeniero a cargo del desarrollo del proyecto.

\subsection{Alcance}
\label{sec:orgaf51da6}

El alcance del test de aceptación se restringirá exclusivamente a la ejecución precisa de instrucciones por parte del emulador, excluyendo la evaluación del funcionamiento de los periféricos que puedan estar emulados.

\subsection{Objetivos}
\label{sec:orga40b0ee}

Los objetivos son:

\begin{itemize}
\item Determinal si el sistema cumple con los requerimientos.
\item Reportar las diferencias entre el comportamiento esperado del sistema y el observado.
\item Generar un ambiente de pruebas automáticas que permita verificar el correcto funcionamiento del emulador a medida que su desarrollo avance.
\end{itemize}


\subsection{Precondiciones}
\label{sec:org5ca5790}

Las precondiciones extenas son:

\begin{itemize}
\item Se dispondrá en todo momento del emulador en su fase de desarrollo.
\item Se tendrá acceso a un servidor GitLab para pruebas automáticas.
\item Se tendrá comunicación con expertos y se podrán realizar pruebas contra un modelo de referencia para clarificar dudas y chequear resultados.
\item El emulador a ser testeado dispondrá de una API bien documentada para su comunicación.
\end{itemize}


\section{Bases del test}
\label{sec:bases_del_test}

Las bases del test consisten en las siguientes especificaciones:
\begin{itemize}
\item Planificación del producto y requerimientos.
\item Especificación de requerimientos de software.
\item Documento de especificación de arquitectura.
\end{itemize}

\section{Estrategia general del test}
\label{sec:estra_general_test}

\subsection{Características de calidad}
\label{sec:caracteristicas_de_calidad}

Las caracterpísticas de calidad para realizar el test:

%%%%%%%%%%%%%%%%%%%%%%%%%%%%%%%%%%%%%%%%%%%%%%%%%%
\begin{table}[h!]
	\centering
	\begin{tabular}{ | m{5cm} | m{5cm} | }
		\hline
		\rowcolor{gray!50} % Coloring the first row
		\textbf{Característica de calidad} & \textbf{Importancia relativa (\%)} \\ \hline
    Funcionalidad & 50 \\ \hline
    Confiabilidad & 25 \\ \hline
    Cambiabilidad & 15 \\ \hline
    Usabilidad & 10 \\ \hline
    \rowcolor{gray!50} % Coloring the first row
		\textbf{TOTAL} & \textbf{100}
    \\ \hline
	\end{tabular}

\end{table}
%%%%%%%%%%%%%%%%%%%%%%%%%%%%%%%%%%%%%%%%%%%%%%%%%%

\begin{itemize}
\item Funcionalidad: El emulador del procesador debe ser capaz de ejecutar las instrucciones del procesador de manera precisa y sin errores, ya que esto es esencial para garantizar la fiabilidad de los resultados de la emulación.
\item Confiabilidad: Debe ser capaz de mantener un rendimiento estable y confiable durante largos períodos de procesamiento continuo, incluso durante horas de ejecución ininterrumpida. El emulador debe ser capaz de gestionar eficientemente los recursos y prevenir cualquier degradación del rendimiento que pueda surgir debido a la carga prolongada, garantizando así una emulación consistente y precisa del procesador.
\item Cambiabilidad: El software debe poder ser fácilmente actualizable y adaptable. La razón de esto es que se trata de un software en desarrollo activo, donde nuevas versiones serán emitidas constantemente.
\item Usabilidad: Se debe tener acceso rápido y sencillo a información relevante sobre el estado del emulador. Se debe poder visualizar de manera clara y accesible los parámetros relacionados con la emulación del procesador, como el estado de sus registros y cualquier mensaje de error o advertencia que pueda surgir durante la operación. Esto facilitará la monitorización y el diagnóstico de posibles problemas, garantizando un uso eficiente y efectivo del emulador.
\end{itemize}


\subsection{Características de calidad de niveles de prueba}
\label{sec:caracteristicas_de_calidad_niveles_de_prueba}

%%%%%%%%%%%%%%%%%%%%%%%%%%%%%%%%%%%%%%%%%%%%%%%%%%
\begin{table}[h!]
	\centering
	\begin{tabular}{ | m{3cm} | m{2.7cm} | m{2.5cm} | m{2.7cm} | m{2.5cm} | }
		\hline
		\rowcolor{gray!50} % Coloring the first row
		\textbf{-} & \textbf{Funcionalidad} & \textbf{Confiabilidad} & \textbf{Cambiabilidad} & \textbf{Usabilidad} \\ \hline
    \textbf{Importancia relativa} & 50\% & 25\% & 15\% & 10\%  \\ \hline
    \textbf{Unit test} & ++ & + &  & +  \\ \hline
    \textbf{Software integration test} & + & ++ & + & +  \\ \hline
    \textbf{Hardware / Software integration test} &  &  &  &   \\ \hline
    \textbf{Acceptance Tests} & ++ & ++ & ++ & ++  \\ \hline
    \textbf{Field test} & + & + & + & +
    \\ \hline
	\end{tabular}

\end{table}
%%%%%%%%%%%%%%%%%%%%%%%%%%%%%%%%%%%%%%%%%%%%%%%%%%

Referencias:
\begin{itemize}
\item ++: La característica de calidad es predominante para el nivel de test.
\item +: La característica de calidad es relevante para el nivel de test.
\item Vacío: La característica de calidad no es relevante para el nivel de test.
\end{itemize}

Se indica a continuación el motivo de la selección de cada tipo de test para cada tipo decaracterística de calidad.

\begin{itemize}
\item Funcionalidad: Esta característica se evalua en todos los niveles de test, ya que es esencial para garantizar el correcto funcionamiento del emulador. Siendo los tests unitarios y de aceptación los más relevantes, ya que el primero permite evaluar el comportamiento de cada componente del emulador de manera aislada, y el segundo permite evaluar el comportamiento del emulador en su conjunto.
\item Confiabilidad: Esta característica se evalua en todos los niveles de test, ya que es esencial para garantizar la confiabilidad del emulador. Siendo los tests de integración de software y de aceptación los más relevantes, ya que ambos permiten verificar el comportamiento del emulador en su conjunto.
\item Cambiabilidad: Esta característica se evalua predominantemente en los tests de aceptación, ya que en ellos se puede garantizar que el emulador puede ser actualizado y adaptado de manera sencilla.
\item Usabilidad: Se evalua en todos los niveles de test, siendo los tests de aceptación los más relevantes, ya que en ellos se puede garantizar que el emulador es fácil de usar y que se puede acceder de manera rápida y sencilla a información relevante sobre su estado.
\end{itemize}


\section{Estrategia por nivel de prueba}
\label{sec:estrategia_por_nivel_de_prueba}

\subsection{División del sistema en subsistemas}
\label{sec:division_del_sistema_en_subsistemas}

A continuación se determinan los subsistemas de software en los que se divide el emulador del procesador. Dichos subsistemas son descritos en mayor detalle en la entrada \textbf{AD.02} de la Tabla~\ref{tab:referencias}.

\begin{itemize}
\item Subsistema A, CPU: Modelo del procesador Leon3, encargado de interpretar cada instrucción y ejecutarla. Dicho subsistema es el núcleo del emulador, y se comunica con el resto de los subsistemas.
\item Subsistema B, Memoria: Encargado de gestionar la memoria emulada del sistema, permitiendo la lectura y escritura de datos. Dicho modelo también permite el mapeo de dispositivos de entrada/salida.
\item Subsistema C, Registros: Encargado de gestionar los registros del procesador, permitiendo la lectura, escritura y volcado de los mismos.
\item Subsistema D, Scheduler y TimeKeeper: Encargado de gestionar el tiempo y la planificación de tareas, permitiendo la emulación de sistemas en tiempo real.
\item Subsistema E, API: Encargado de su instanciación, configuración y de gestionar su comunicación con otros dispositivos, permitiendo el control del mismo.
\end{itemize}

\subsection{Importancia relativa de los subsistemas}
\label{sec:importancia_relativa_de_los_subsistemas}

%%%%%%%%%%%%%%%%%%%%%%%%%%%%%%%%%%%%%%%%%%%%%%%%%%
\begin{table}[h!]
	\centering
	\begin{tabular}{ | m{5cm} | m{5cm} | }
		\hline
		\rowcolor{gray!50} % Coloring the first row
		\textbf{Subsistema} & \textbf{Importancia relativa (\%)} \\ \hline
    CPU & 30 \\ \hline
    Memoria & 15 \\ \hline
    Registros & 30 \\ \hline
    Scheduler \& TimeKeeper & 10 \\ \hline
    API & 15 \\ \hline
    \rowcolor{gray!50} % Coloring the first row
		\textbf{TOTAL} & \textbf{100}
    \\ \hline
	\end{tabular}

\end{table}
%%%%%%%%%%%%%%%%%%%%%%%%%%%%%%%%%%%%%%%%%%%%%%%%%%


\subsection{Determinación de  la importancia  de test por combinaciones de subsistema y características de calidad}
\label{sec:determinacion_importancia_test}

%%%%%%%%%%%%%%%%%%%%%%%%%%%%%%%%%%%%%%%%%%%%%%%%%%
\begin{table}[h!]
	\centering
	\begin{tabular}{ | m{3cm} | m{2cm} | m{2cm} | m{2cm} | m{2cm} | m{2cm} | }
		\hline
		\rowcolor{gray!50} % Coloring the first row
		\textbf{-} & \textbf{Sub. A} & \textbf{Sub. B} & \textbf{Sub. C} & \textbf{Sub. D} & \textbf{Sub.E} \\ \hline
    \textbf{Importancia relativa} & 30\% & 15\% & 30\% & 10\% & 15\%  \\ \hline
    \textbf{Funcionalidad} & ++ & + & ++ &  & + \\ \hline
    \textbf{Confiabilidad} & ++ & + & ++ & ++ & \\ \hline
    \textbf{Cambiabilidad} &  &  &  & & ++ \\ \hline
    \textbf{Usabilidad} & + & + & + & + & ++
    \\ \hline
	\end{tabular}

\end{table}
%%%%%%%%%%%%%%%%%%%%%%%%%%%%%%%%%%%%%%%%%%%%%%%%%%

Referencias:
\begin{itemize}
\item ++: La característica de calidad es predominante para el subsistema.
\item +: La característica de calidad es relevante para el subsistema.
\item Vacío: La característica de calidad no es relevante para el subsistema.
\end{itemize}


\end{document}
